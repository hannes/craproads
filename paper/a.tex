\documentclass[preprint,12pt]{elsarticle}

\usepackage{graphics}
\usepackage{float}
\usepackage{epsfig}
\usepackage{dcolumn}
\usepackage[implicit=false]{hyperref}
\usepackage{amsthm}
\usepackage{algorithmicx}
\usepackage{algpseudocode}
\usepackage{latexsym}
\usepackage{algorithm}
\usepackage{caption}
\usepackage{multirow}
\usepackage{stmaryrd}
\usepackage{amssymb}
\usepackage{latexsym}
\usepackage{color}
\usepackage{rotating}
\usepackage{tabularx}


\journal{Lorentz Center Meeting}


\newcommand{\mi}[1]{\ensuremath{\mathit{#1}}}
\theoremstyle{definition}
\newtheorem{definition}{Definition}
\newcommand{\todo}[1]{\textcolor{red}{#1}}


\begin{document}
\begin{frontmatter}

\title{Opportunistic sensing to improve bike traffic}

\author{RWS group}
%\ead{e.kaldeli@rug.nl}
%\address{University of Groningen}


%% \title{Title\tnoteref{label1}}
%% \tnotetext[label1]{}
%% \author{Name\corref{cor1}\fnref{label2}}
%% \ead{email address}
%% \ead[url]{home page}
%% \fntext[label2]{}
%% \cortext[cor1]{}
%% \address{Address\fnref{label3}}
%% \fntext[label3]{}


%% use optional labels to link authors explicitly to addresses:
%% \author[label1,label2]{<author name>}
%% \address[label1]{<address>}
%% \address[label2]{<address>}





%\begin{abstract}
%An executive summary should go here.
%\end{abstract}



%\begin{keyword}
%\end{keyword}

\end{frontmatter}


\section{Context}
Rijkswaterstaat from the Ministry of Infrastructure and the
Environment is interested to investigate how citizens, for instance
while cycling, can provide information about the infrastructure they
are using without significant involvement and without increasing the
risk of accidents. Given the current context of the Netherlands that
indicates that 4 million citizens use smartphones while traveling and
that 50\% of Dutch bikers read social media messages while biking, one
may take advantage of the sensing capabilities of smartphones while
people are active on their bikes. The challenges are, on the one hand, to understand
what is possible to infer about the infrastructure from using the
sensing capabilities of a smartphone and, on the other hand, to design
an app which is easy to use, doesn't endanger the user while biking,
is appealing and engaging to employ and guarantees sufficient privacy
to the end-user. 


\section{State of the art}

The pervasiveness of smart devices has open a number of unprecedented
opportunities for what falls under the name of {\em
  crowdsensing}~\cite{gan:mob11}. Modern smartphones have a wide
pletora of sensing interfaces which make interaction with the physical
environment easy and information rich, for instance, a modern iPhone
has: three-axis gyro, accelerometer, magnetometer, GPS, proximity
sensor, ambient light sensor, two high resolution cameras, a
microphone and high-frequency networking interfaces. These have been
exploited for a number of pattern recognition taks,
e.g.~\cite{fuj:iph10}. The specific issue of identifying the quality
of bike paths using dedicated sensors on board of bicycles has been
proposed in~\cite{eis:bik07} while possible social interactions of
bikers have been investigated
in~\cite{red:bik10}. In~\cite{eri:pot08}, the authors consider vehicle
onboard sensing to measure road quality with dedicated solutions.



\section{Problem definition}

\subsection{Infrastructure quality}

One has to first define what are relevant parameters that influence
the infrastructure quality, in our case, bike paths. The relevant
elements appear to be, in order of relevance,
\begin{enumerate}
\item presence of major potholes or other large size obstacles
\item coarseness/smoothness of the biking surface (in relation to the
  material used to create the infrastructure and its age)
\item weather conditions (wetness, ice, snow, in particular)
\item path traffic 
\item neighboring/interfering car traffic
\item quality of air (e.g., CO2 levels)
\item illumination
\end{enumerate}


\subsection{Sensing}

To measure the quality of the infrastructure just listed, we plan to
consider using a smartphone. Precise design choices can be defined
only after experimentation (see Pilot project section). We consider
the following relevant assumptions/choices.
\begin{description}
\item[Position of the phone.] Placing the phone on the frame of the
  bike or handle appears to make the sensing easier and more
  consistent across cyclist, though might be harder for the user to
  accept and will require extra equipment.
\item[Accelerometer.] This sensor is essential, but might not provide
  enough information.
\item[GPS.] Necessary for path tracking, reporting problems on
  specific location, report average speeds on given tracks.
\item[Gyroscope.] Very useful, especially if the phone is on the
  person and not on the bike. It may also help detect hazardous
  situations, such as the fall of the biker.
\item[Light sensor.] Might help in darkness conditions, if the phone
  is on the bike.
\item[Magnetometer.] Might help to detect the amount of surrounding
  cars.
\item[Camera.] Very useful, but putting too much computational and
  battery stress on the phone, therefore ignored in the following.
\end{description}


\subsection{Distribution architecture: data management and privacy}

There is an important issue on where the data processing occurs and
where raw and inferred data is stored. Another related relevant
problem is on when is data transferred from the mobile devices to the
data aggregation points. We foresee a solution in which raw data is
processed locally on the smartphone, inferred data is transmitted to
the RWS servers when the user docks in his base station (e.g., work or
home pc), unless a dangerous situation requires immediate data transmission
(major infrastructure problem or accident to the cyclist). Also the
storage of the data implies important privacy concerns. The sensed
data includes highly sensitive information on the location and
behaviors of the end user. Storing on the mobile phone is neither
feasible nor useful. Storing centrally anonymously seems important,
though not giving sufficient privacy guarantees. On the other hand,
having user profiles related to sensed data, would highly increase the
quality of the acquired data.

\begin{center}
ADD FIGURE FROM WHITEBOARD
\end{center}


\subsection{User interaction}

The user should minimize the interaction with the app. If the app
provides extra features (e.g., traffic information, turn by turn
routing, biometrics), these should be presented to the user without
distracting him/her from the driving. The opportunistic sensing should
be done in the background and user confirmation should occur as an ex
post notification. For instance, when docking the user is prompted to
confirm that a strong acceleration was related to a major pothole in
the pavement. 


\subsection{Incentive scheme}

One has to design an effective incentive scheme for the end user. Some
will be willing to help the community independently of any personal
advantage. We fear that though these kind of users will be a
minority. Other incentives might come from the certainty that
reporting infrastructural problems will improve a route that the user
frequently follows. More short term high reward incentives, might come
from having a free navigation and traffic information system be given
``for free'' with the infrastructure monitoring app. A quantitative
study of what works is necessary to establish an effective marketing
strategy for the app.


\section{Scenarios}


\section{Pilot project to check feasibility}



\section{The way ahead}

Give us a lot of money and we can solve anything.



\bibliographystyle{apalike}
\bibliography{rws}
\end{document}






